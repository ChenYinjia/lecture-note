\begin{introduction}
    \item 群的基本概念
\end{introduction}

\section{群的基本概念}
\subsection{群,群的阶,子群,陪集}
    何谓群论?群论就是研究集合的结构特征及其生成规律的学科。也就是说,群就是具备了一些结构的集合,这些结构是科学家们在物理世界中提取出来的具有共性的性质。在这里我们直接给出群的定义:
    \begin{definition}{群的定义}{group}
        设$G$是一些元素的集合,记为:$G=\{\dots,g,\dots\}$,在G中定义了一种运算\footnote{根据线性代数的内容,我们知道运算实际上就是一种类似$(a,b)\rightarrow c$的映射,其中$(a,b)$是笛卡尔积},称为乘法,乘法满足以下性质:
        封闭性:任意两个元素的乘积仍然在G中,即:$\forall g_1,g_2\in G,\exists g\in G$,使得$g_1g_2=g$;\\
        结合律:$\forall g_1,g_2,g_3\in G$,都有$(g_1g_2)g_3=g_1(g_2g_3)$;\\
        单位元:存在唯一一个元素$e$,使得$\forall g\in G,ge=eg=g$;\\
        逆元:$\forall g\in G$,存在唯一一个元素$g^{-1}\in G$,使得$gg^{-1}=g^{-1}g=e$
        
        此时我们称G是一个群。
    \end{definition}
\subsection{类,不变子群,商群}

\subsection{同态,同构}
\subsection{}