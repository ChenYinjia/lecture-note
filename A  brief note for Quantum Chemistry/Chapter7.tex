\begin{introduction}
    \item 变分法的基本原理
    \item 例子:谐振子与四次势
    \item 线性变分法
\end{introduction}
从本章开始,我们将会介绍两种Schrodinger方程的近似方法,即微扰理论和变分法。微扰理论基于已经有精确解的Schrodinger方程,我们将实际体系的Hamiltonian拆分成精确解对应的Hamiltonian和所谓的微扰Hamiltonian,后者在量级上是比前者作用小很多的。这种处理问题的方式是非常“物理”的:即我们突出物理模型的主要性质,在此基础上,我们再引入近似项,由此来考虑物理模型更“精细”的结构。但是问题也是很明显的,即必须要将体系的Hamiltonian分解为某个可以精确求解的Hamiltonian和某个微扰Hamiltonian的和。但是在很多体系中是难以拆分的;同时在后面我们可以发现,微扰法在考虑高阶波函数与能量近似时计算非常复杂。本节介绍的变分法是量子化学中更为常见的近似方法。
\section{变分法的基本原理}
所谓变分法,简单来说就是求解某一个泛函的极值。求解泛函的方法有很多种,在量子化学中,我们通常采用一种叫Ritzs变分法的方法。这类变分法的基本思想基于以下不等式:任意给定一个well-behaved的态$|\phi\rangle$(我们称之为试探变分函数,trial variation function),都有\footnote{可以发现,试探函数$|\phi\rangle$的要求不高,只需要满足体系的边界条件即可,因此如果选择得当,其收敛速度远远比精确波函数快}:
\begin{equation}\label{equ9:variationequ}
    \frac{\langle \phi|\hat{H}|\phi\rangle}{\langle \phi|\phi\rangle}\geq E_0
\end{equation}

其中$\hat{H}$是体系的Hamiltonian,$E_0$是体系的基态能量。可以看出对于态空间任意的态,其能量的平均值一定不小于基态能量,下面给出证明。

设体系的本征方程为:
\begin{equation}
    \hat{H}|\psi_n\rangle=E_n|\psi_n\rangle
\end{equation}

自然,$|\phi\rangle$可以写成本征态的线性组合:
\begin{equation}
    |\phi\rangle=\sum_n c_n|\psi_n\rangle
\end{equation}

于是则有
\begin{align}
    \begin{split}
        \langle \phi|\hat{H}|\phi\rangle=&\sum_{m,n}c_m^*c_n\langle\phi_m|\hat{H}|\phi_n\rangle\\
        =& \sum_{m,n}c_mc_n E_n \delta_{mn}\\
        =& \sum_n |c_n|^2 E_n \\
        \geq& \sum_n |c_n|^2 E_1 \\
        =& E_1\langle \phi|\phi\rangle
    \end{split}
\end{align}

移项即可得\ref{equ9:variationequ}式。进一步,如果$|\phi\rangle$含有参数$\alpha$,那么显然最接近基态能量$E_0$的能量平均值$\langle E(\alpha^*)\rangle$应该满足:
\begin{equation}
    \left.\frac{\partial \langle E(\alpha)\rangle}{\partial \alpha_i}\right|_{\alpha=\alpha^*}=0
\end{equation}

这样我们就可以得到该函数下最好的基态描述。如果试探函数选择得当,那么变分法可以得到非常精确的结果。
\section{一个简单的例子}

\section{线性变分法}
上面是一些简单的例子,在实际研究中,试探函数的建立没有那么容易。由于我们希望试探函数的建立是基于一定的物理的,于是往往会将试探函数$|\phi\rangle$用多个线性无关的实函数$|f_i\rangle$展开来表示,即:
\begin{align}
    \begin{split}
        |\phi\rangle=\sum_{i=1}^n c_i|f_i\rangle
    \end{split}
\end{align}

上式代入\ref{equ9:variationequ},可得:
\begin{align}\label{equ9:linearvariation}
    \begin{split}
        \langle E\rangle=&\frac{\langle\phi|\hat{H}|\phi\rangle}{\langle\phi|\phi\rangle}\\
        =&\frac{\sum_{i,j}^nc_ic_j\langle f_i|\hat{H}|f_j\rangle}{\sum_{i,j}^n c_ic_j\langle f_i|f_j\rangle}\\
        \Rightarrow&\langle E\rangle\sum_{i,j}^n c_ic_jS_{ij}=\sum_{i,j}^n c_ic_jH_{ij}
    \end{split}
\end{align}

可以看到,$\langle E\rangle$是含有参数$\{c_i\}$的,于是需要满足:
\begin{equation}
    \frac{\partial \langle E\rangle}{\partial c_k}=0,k=1,2,\dots,n
\end{equation}

于是我们考虑对\ref{equ9:linearvariation}式进行隐函数求偏导:
\begin{equation}\label{equ9:A}
     \frac{\partial\langle E\rangle}{\partial c_k}\sum_{i,j}^n c_ic_jS_{ij}+\langle E\rangle\frac{\partial}{\partial c_k}\Big(\sum_{i,j}^n c_ic_jS_{ij}\Big)=\frac{\partial}{\partial c_k}\Big(\sum_{i,j}^n c_ic_jH_{ij}\Big)
\end{equation}

其中$S_{ij}=\langle f_i|f_j\rangle$,称为重叠积分\footnote{为了保持连贯性,因此我这里仍然使用Dirac符号,但是实际我们计算的时候是采用积分形式的};$H_{ij}=\langle f_i|\hat{H}|f_j\rangle$

由于$S_{ij}$是常数,因此上式中的几项都可以化简:
\begin{align}
    \begin{split}
        \frac{\partial}{\partial c_k}\Big(\sum_{i,j}^n c_ic_jS_{ij}\Big)=&\sum_{i,j}^n\frac{\partial}{\partial c_k}( c_ic_j)S_{ij}\\
        =&\sum_{i,j}^n(\frac{\partial c_i}{\partial c_k}c_j+\frac{\partial c_j}{\partial c_k}c_i)S_{ij}
    \end{split}
\end{align}

由于:
\begin{equation}
    \frac{\partial c_i}{\partial c_j}=\delta_{ij}
\end{equation}

于是:
\begin{equation}
    \sum_{i,j}^n(\frac{\partial c_i}{\partial c_k}c_j+\frac{\partial c_j}{\partial c_k}c_i)S_{ij}=\sum_j c_jS_{kj}+\sum_i c_i S_{ik}
\end{equation}

由于$S_{ij}$是一个内积形式,我们根据内积定义,结合内积是实数,可以知道:
\begin{equation}
    \langle f_i|f_j\rangle=(\langle f_j|f_i\rangle)^*=\langle f_j|f_i\rangle
\end{equation}

即:$S_{ij}=S_{ji}$,于是上式可以变为:
\begin{equation}
    \frac{\partial}{\partial c_k}\Big(\sum_{i,j}^n c_ic_jS_{ij}\Big)=\sum_j c_jS_{kj}+\sum_i c_i S_{ik}=2\sum_i c_i S_{ik}
\end{equation}

同理:
\begin{equation}
    \frac{\partial}{\partial c_k}\Big(\sum_{i,j}^n c_ic_jH_{ij}\Big)=\sum_j c_jS_{kj}+\sum_i c_i H_{ik}=2\sum_i c_i H_{ik}
\end{equation}

于是,\ref{equ9:A}式可以化简为:
\begin{equation}\label{equ9:B}
    \sum_i c_iH_{ik}-\langle E\rangle\sum_i c_iS_{ik}=0
\end{equation}

可以看到,上式是一个关于$\{c_i\}$的齐次线性方程组,我们知道,齐次线性方程组有非平凡解当且仅当系数行列式为0,即满足:
\begin{equation}
    \det(H-\langle E\rangle S)=0
\end{equation}

通过求解行列式我们可以得到$\langle E\rangle$的前n个解:$\langle E\rangle_1,\langle E\rangle_2,\dots,\langle E\rangle_n$,可以证明体系的能量与$\langle E\rangle$的关系为:
\begin{equation}
    E_1\leq \langle E\rangle_1, E_2\leq \langle E\rangle_2,\dots, E_n\leq \langle E\rangle_n
\end{equation}

因此,线性变分法的结果就是给出了体系前n级能量的上界(upper bound)。如果我们想要获得体系的更多级能量,在构造试探函数的时候就要用更多数量的函数$|f_i\rangle$展开,同时更完备的函数集可能会提高求解能量的精确性,但是可惜的是,由于计算机算力的限制,我们往往只能使用有限的基函数来代表完备集。

特别的,如果$\{|f_i\rangle\}$是正交的,那么$S_{ij}=\langle f_i|f_j\rangle=\delta_{ij}$。因此\ref{equ9:B}式可以表示为:
\begin{equation}
     \sum_i c_iH_{ik}-\langle E\rangle c_k=0
\end{equation}

这也是一个线性齐次方程组,我们可以将其展开写成:
\begin{align}
    \begin{split}
        H_{11}c_1+H_{12}c_2+\dots+H{1n}c_n=&\langle E\rangle c_1\\
        H_{21}c_1+H_{22}c_2+\dots+H{2n}c_n=&\langle E\rangle c_2\\
        \dots\dots\dots\dots\dots\\
        H_{n1}c_1+H_{n2}c_2+\dots+H{nn}c_n=&\langle E\rangle c_n
    \end{split}
\end{align}

可以明显看出,这个方程组可以写成本征方程的形式,即:
\begin{equation}
    Hc=\langle E\rangle c
\end{equation}

其中$H=\begin{pmatrix} H_{11}&H_{12}&\dots&H_{1n}\\H_{21}&H_{22}&\dots&H_{2n}\\ \dots&\dots&\dots&\dots\\H_{n1}&H_{n2}&\dots&H_{nn}\end{pmatrix},c=\begin{pmatrix}c_1\\c_2\\ \vdots\\c_n\end{pmatrix}$

对于计算机来说,更好的求解方程组的手段是求解矩阵方程。因此我们换一种方式理解线性变分法的求解。令上述本征方程的n个本征值为$W_1,W_2,\dots,W_n$,本征态为$c^{i}$\footnote{注意,这是一个列矢量},那么本征方程自然可以表达为:
\begin{equation}
    Hc^{(i)}=W_ic^{(i)},i=1,2,\dots,n
\end{equation}

显然,我们可以通过一个矩阵方程将上面n个本征方程组合并:
\begin{equation}
    HC=CW
\end{equation}

其中:
$C=\begin{pmatrix}c_{1}^{(1)}&c_{1}^{(2)}&\dots&c_{1}^{(n)}\\c_{2}^{(1)}&c_{2}^{(2)}&\dots&c_{2}^{(n)}\\ \dots&\dots&\dots&\dots\\c_{n}^{(1)}&c_{n}^{(2)}&\dots&c_{n}^{(n)}\end{pmatrix},W=\begin{pmatrix}W_1&\quad&\quad&\quad\\ \quad&W_2&\quad&\quad \\ \quad&\quad&\ddots&\quad\\ \quad &\quad&\quad&W_n \end{pmatrix}$。组成矩阵C的向量集$\{c^{(i)}\}$是正交的,因此C一定可逆。如果H是Hermitian,则C是一定是幺正矩阵,那么矩阵方程就可以看作是将H矩阵对角化的过程:
\begin{equation}
    W=C^\dagger H C
\end{equation}

实际上,求解行列式的本质在于求解多项式的根,其坏处在于,如果对多项式的系数作微小的改变,最后的结果会相差很大。这就需要计算机对于多项式根的求解有非常高的灵敏度;而矩阵的对角化问题不仅抗扰动能力强,同时在数值计算中已经有比较成熟的处理方法了。因此,目前我们主要采取后者作为分析的主要方法。